\documentclass[11pt]{letter}
\usepackage[margin=1in]{geometry}
\usepackage{hyperref}

\signature{Ian Todd\\Sydney Medical School\\University of Sydney}
\address{Ian Todd\\Sydney Medical School\\University of Sydney\\Sydney, NSW, Australia\\itod2305@uni.sydney.edu.au}

\begin{document}

\begin{letter}{Dr. Abir Igamberdiev\\Editor-in-Chief\\BioSystems}

\opening{Dear Dr. Igamberdiev,}

I am pleased to submit ``The Geometry of Biological Shadows: Quantifying Topological Aliasing in High-Dimensional Systems'' for consideration in \textit{BioSystems} as a computational companion to my recently published paper (Todd, 2025; vol.\ 258, 105608).

\textbf{Context:}

The earlier paper argued theoretically that Popperian falsifiability faces structural limits in high-dimensional biological systems. That work remained qualitative. This manuscript provides the computational framework to \textit{measure} those limits.

\textbf{The Central Contribution:}

We introduce the ``Topological Aliasing Rate''---the first standardized metric for quantifying how much 2D projections (t-SNE, UMAP) misrepresent high-dimensional neighborhood relationships. Our key empirical finding: across four standard single-cell benchmark datasets (90,300 cells total), \textbf{75.5\% of apparent neighbors in 2D projections were NOT neighbors in the original high-dimensional space}.

This is not a property of any particular dataset. It is a geometric consequence of projecting $D_{\text{sys}} \approx 10$--40 dimensions into $D_{\text{obs}} = 2$.

\textbf{The Deliverable:}

The accompanying \texttt{falsifiability} Python library (pip-installable, archived on Zenodo: DOI 10.5281/zenodo.17791874) enables any researcher to compute aliasing rates on their own data. We provide practical calibration guidelines:
\begin{itemize}
\item Aliasing $<30\%$: Visual clusters likely reflect high-dimensional topology
\item Aliasing 30--60\%: Clusters should be interpreted with caution
\item Aliasing $>60\%$: 2D is for visualization only; quantitative analysis must occur in high-dimensional space
\end{itemize}

\textbf{Why BioSystems:}

This work directly extends the epistemological framework of my earlier \textit{BioSystems} paper, operationalizing philosophical arguments into measurable quantities. The audience that engaged with ``The Limits of Falsifiability'' is precisely the audience for this computational companion. The work bridges systems biology, bioinformatics, and philosophy of science---the intersection that defines \textit{BioSystems}.

Thank you for your consideration.

\closing{Sincerely,}

\end{letter}

\end{document}
